%\part{Metodologia}



\chapter[Metodologia]{Metodologia}
\addcontentsline{toc}{chapter}{Metodologia}
\label{sec:experimental-design}

\setcounter{footnote}{23}


Os principais objetivos desta avaliação de desempenho são: (a) determinar o \textit{overhead} da ferramenta Bundler em ambientes virtualizados; (b) verificar qual recurso é mais crucial no ambiente virtualizado, CPU, memória RAM e/ou tempo de processamento; e (c) verificar o ganho em migrar a aplicação para o ambiente de nuvem.

\subsection{Cenários}
\label{subsec:cenarios}

Neste trabalho foram feitas avaliações em dois cenários. O primeiro cenário consiste em uma máquina com configurações simples, e o segundo  O LXC foi utilizado pois utiliza recursos do Kernel do Linux para criar os \textit{containers} \cite{dua2014virtualization}.

\subsection{Metodologia}
\label{subsec:metodologia}

Para alcançar os objetivos definidos neste trabalho, adaptou-se a metodologia utilizada por \cite{sousa2012evaluating} para executar os experimentos, baseado em quatro principais atividades (Figura \ref{fig:methodology}):


\begin{figure}[h!]
\centering
\includegraphics[width=1\columnwidth]{imagens/metodologia.png}
\caption{Metodologia de execução dos experimentos (adaptado de \cite{sousa2012evaluating})}
\label{fig:methodology}
\end{figure}

