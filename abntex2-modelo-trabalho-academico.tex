%% abtex2-modelo-trabalho-academico.tex, v-1.7.1 laurocesar
%% Copyright 2012-2013 by abnTeX2 group at http://abntex2.googlecode.com/ 
%%
%% This work may be distributed and/or modified under the
%% conditions of the LaTeX Project Public License, either version 1.3
%% of this license or (at your option) any later version.
%% The latest version of this license is in
%%   http://www.latex-project.org/lppl.txt
%% and version 1.3 or later is part of all distributions of LaTeX
%% version 2005/12/01 or later.
%%
%% This work has the LPPL maintenance status `maintained'.
%% 
%% The Current Maintainer of this work is the abnTeX2 team, led
%% by Lauro César Araujo. Further information are available on 
%% http://abntex2.googlecode.com/
%%
%% This work consists of the files abntex2-modelo-trabalho-academico.tex,
%% abntex2-modelo-include-comandos and abntex2-modelo-references.bib
%%

% ------------------------------------------------------------------------
% ------------------------------------------------------------------------
% abnTeX2: Modelo de Trabalho Academico (tese de doutorado, dissertacao de
% mestrado e trabalhos monograficos em geral) em conformidade com 
% ABNT NBR 14724:2011: Informacao e documentacao - Trabalhos academicos -
% Apresentacao
% ------------------------------------------------------------------------
% ------------------------------------------------------------------------

\documentclass[
	% -- opções da classe memoir --
	12pt,				% tamanho da fonte
	openright,			% capítulos começam em pág ímpar (insere página vazia caso preciso)
	oneside,			% para impressão em verso e anverso. Oposto a oneside
	a4paper,			% tamanho do papel. 
	% -- opções da classe abntex2 --
	%chapter=TITLE,		% títulos de capítulos convertidos em letras maiúsculas
	%section=TITLE,		% títulos de seções convertidos em letras maiúsculas
	%subsection=TITLE,	% títulos de subseções convertidos em letras maiúsculas
	%subsubsection=TITLE,% títulos de subsubseções convertidos em letras maiúsculas
	% -- opções do pacote babel --
	english,			% idioma adicional para hifenização
	french,				% idioma adicional para hifenização
	spanish,			% idioma adicional para hifenização
	brazil,				% o último idioma é o principal do documento
	]{abntex2}


% ---
% PACOTES
% ---

% ---
% Pacotes fundamentais 
% ---
\usepackage{cmap}				% Mapear caracteres especiais no PDF
\usepackage{lmodern}			% Usa a fonte Latin Modern			

\usepackage{pdfpages} %incluir PDFs

%\usepackage{times}
%\usepackage{pslatex} %para usar times new roman

\usepackage{helvet} %arial
\renewcommand{\familydefault}{\sfdefault} % arial

%\usepackage[T1]{fontenc}		% Selecao de codigos de fonte.
\usepackage[utf8]{inputenc}		% Codificacao do documento (conversão automática dos acentos)
\usepackage{lastpage}			% Usado pela Ficha catalográfica
\usepackage{indentfirst}		% Indenta o primeiro parágrafo de cada seção.
\usepackage{color}				% Controle das cores
\usepackage{graphicx}			% Inclusão de gráficos
% ---
		
% ---
% Pacotes adicionais, usados apenas no âmbito do Modelo Canônico do abnteX2
% ---
\usepackage{lipsum}				% para geração de dummy text
% ---

% ---
% Pacotes de citações
% ---
\usepackage[brazilian,hyperpageref]{backref}	 % Paginas com as citações na bibl


\usepackage[alf,abnt-etal-cite=2]{abntex2cite}	% Citações padrão ABNT

% Adicionando as configurações do template da UPE Caruaru
\usepackage{conf/upe-caruaru}

\usepackage{chngcntr}

% --- 
% CONFIGURAÇÕES DE PACOTES
% --- 

% ---
% Configurações do pacote backref
% Usado sem a opção hyperpageref de backref
\renewcommand{\backrefpagesname}{Citado na(s) página(s):~}
% Texto padrão antes do número das páginas
\renewcommand{\backref}{}
% Define os textos da citação
\renewcommand*{\backrefalt}[4]{
	\ifcase #1 %
		Nenhuma citação no texto.%
	\or
		Citado na página #2.%
	\else
		Citado #1 vezes nas páginas #2.%
	\fi}%

% ---


% ---
% Informações de dados para CAPA e FOLHA DE ROSTO
% ---
\titulo{Geolocalização Indoor: Sistema de publicidade digital direcionada utilizando tratamento de posição e correção de rota em ambiente interno}
\autor{Dayvid Clementino da Silva}
\local{Caruaru}
\data{2017}
\orientador{Jorge Cavalcanti Fonseca}
%\coorientador{Equipe \abnTeX}
\instituicao{%
  Universidade de Pernambuco - UPE
  \par
  Campus Caruaru
  \par
  Graduação}
\tipotrabalho{Monografia}
% O preambulo deve conter o tipo do trabalho, o objetivo, 
% o nome da instituição e a área de concentração 
\preambulo{Monografia apresentada junto ao Curso de Sistemas de Informação da Universidade de Pernambuco, campus Caruaru, como requisição parcial à obtenção do título de Bacharel.}
% ---


% ---
% Configurações de aparência do PDF final

% alterando o aspecto da cor azul
\definecolor{blue}{RGB}{41,5,195}

% informações do PDF
\makeatletter
\hypersetup{
     	%pagebackref=true,
		pdftitle={\@title}, 
		pdfauthor={\@author},
    	pdfsubject={\imprimirpreambulo},
	    pdfcreator={LaTeX with abnTeX2},
		pdfkeywords={abnt}{latex}{abntex}{abntex2}{trabalho acadêmico}, 
		colorlinks=true,       		% false: boxed links; true: colored links
    	linkcolor=black,          	% color of internal links
    	citecolor=black,        		% color of links to bibliography
    	filecolor=magenta,      		% color of file links
		urlcolor=blue,
		bookmarksdepth=4
}
\makeatother
% --- 

% --- 
% Espaçamentos entre linhas e parágrafos 
% --- 

% O tamanho do parágrafo é dado por:
\setlength{\parindent}{1.3cm}

% Controle do espaçamento entre um parágrafo e outro:
\setlength{\parskip}{0.2cm}  % tente também \onelineskip

% ---
% compila o indice
% ---
\makeindex
% ---

% SETCOUNT FOOTNOTE: Revisão da Literatura e Metodologia (Temporário)


% ----
% Início do documento
% ----
\begin{document}

% Retira espaço extra obsoleto entre as frases.
\frenchspacing 

% ----------------------------------------------------------
% ELEMENTOS PRÉ-TEXTUAIS
% ----------------------------------------------------------
% \pretextual


%------------------------------------------------
% Capa

\imprimircapa
% -----------------------------------------------



%------------------------------------------------
% Folha de rosto
% (o * indica que haverá a ficha bibliográfica)

\imprimirfolhaderosto*
% -----------------------------------------------



%------------------------------------------------
% Inserir a ficha bibliografica

%
% Isto é um exemplo de Ficha Catalográfica, ou ``Dados internacionais de
% catalogação-na-publicação''. Você pode utilizar este modelo como referência. 
% Porém, provavelmente a biblioteca da sua universidade lhe fornecerá um PDF
% com a ficha catalográfica definitiva após a defesa do trabalho. Quando estiver
% com o documento, salve-o como PDF no diretório do seu projeto e substitua todo
% o conteúdo de implementação deste arquivo pelo comando abaixo:
%
% \begin{fichacatalografica}
%     \includepdf{fig_ficha_catalografica.pdf}
% \end{fichacatalografica}
\begin{fichacatalografica}
	\vspace*{\fill}					% Posição vertical
	\hrule							% Linha horizontal
	\begin{center}					% Minipage Centralizado
	\begin{minipage}[c]{12.5cm}		% Largura
	
	\imprimirautor
	
	\hspace{0.5cm} \imprimirtitulo  / \imprimirautor. --
	\imprimirlocal, \imprimirdata-
	
	\hspace{0.5cm} \pageref{LastPage} p. : il. (algumas color.) ; 30 cm.\\
	
	\hspace{0.5cm} \imprimirorientadorRotulo~\imprimirorientador\\
	
	\hspace{0.5cm}
	\parbox[t]{\textwidth}{\imprimirtipotrabalho~--~\imprimirinstituicao,
	\imprimirdata.}\\
	
	\hspace{0.5cm}
		1. Palavra-chave1.
		2. Palavra-chave2.
		I. Orientador.
		II. Universidade xxx.
		III. Faculdade de xxx.
		IV. Título\\ 			
	
	\hspace{8.75cm} CDU 02:141:005.7\\
	
	\end{minipage}
	\end{center}
	\hrule
\end{fichacatalografica}
%------------------------------------------------



%------------------------------------------------
% Inserir errata

%\begin{errata}
%Elemento opcional da \citeonline[4.2.1.2]{NBR14724:2011}. Exemplo:
teste
\vspace{\onelineskip}

FERRIGNO, C. R. A. \textbf{Tratamento de neoplasias ósseas apendiculares com
reimplantação de enxerto ósseo autólogo autoclavado associado ao plasma
rico em plaquetas}: estudo crítico na cirurgia de preservação de membro em
cães. 2011. 128 f. Tese (Livre-Docência) - Faculdade de Medicina Veterinária e
Zootecnia, Universidade de São Paulo, São Paulo, 2011.

\begin{table}[htb]
\center
\footnotesize
\begin{tabular}{|p{1.4cm}|p{1cm}|p{3cm}|p{3cm}|}
  \hline
   \textbf{Folha} & \textbf{Linha}  & \textbf{Onde se lê}  & \textbf{Leia-se}  \\
    \hline
    1 & 10 & auto-conclavo & autoconclavo\\
   \hline
\end{tabular}
\end{table}

\end{errata}
%------------------------------------------------



%------------------------------------------------
% Inserir folha de aprovação

%
% Isto é um exemplo de Folha de aprovação, elemento obrigatório da NBR
% 1472pt4/2pt011 (seção 4.2pt.1.3). Você pode utilizar este modelo até a aprovação
% do trabalho. Após isso, substitua todo o conteúdo deste arquivo por uma
% imagem da página assinada pela banca com o comando abaixo:
%
% \includepdf{folhadeaprovacao_final.pdf}
%


\begin{folhadeaprovacao}

\noindent Monografia de Graduação apresentada por \textbf{Guto Leoni Santos} do Curso de Gradua\-ção em Sistemas de Informação da Universidade de Pernambuco, campus Caruaru, sob o título “\textbf{Avaliação de desempenho de tecnologias de virtualização para execução de aplicações de reconstrução 3D em ambientes de nuvem}”, orientada pelo Profa. \textbf{Patricia Takako Endo} e aprovada pela Banca Examinadora formada pelos professores:\vspace{4cm}



\begin{flushright}

\rule{300pt}{2pt}

Thiago Gomes Rodrigues

Faculdade Boa Viagem\vspace{1cm} 


%\rule{300pt}{2pt}

%Prof. Nome do Convidado2 ou Co-orientador

%Depto / Instituição\vspace{1cm} 

\rule{300pt}{2pt}

Profa. Dra. Patricia Takako Endo

Universidade de Pernambuco\vspace{3cm} 

\end{flushright}

\begin{flushleft}

Visto e permitida a impressão.\newline
Caruaru, 09 de fevereiro de 2017.\vspace{1cm} 

\rule{300pt}{2pt}

\textbf{Profa. Dra. Patricia Takako Endo}\newline
Coordenadora do Curso de Bacharelado em Sistemas de\newline Informação da Universidade de Pernambuco - Campus Caruaru.
\end{flushleft}

  
\end{folhadeaprovacao}








\begin{comment}

  \begin{center}
    {\ABNTEXchapterfont\large\imprimirautor}

    \vspace*{\fill}\vspace*{\fill}
    \begin{center}
      \ABNTEXchapterfont\bfseries\Large\imprimirtitulo
    \end{center}
    \vspace*{\fill}
    
    \hspace{.45\textwidth}
    \begin{minipage}{.5\textwidth}
        \imprimirpreambulo
    \end{minipage}%
    \vspace*{\fill}
   \end{center}
        
   Trabalho aprovado. \imprimirlocal, 2pt4 de novembro de 2pt012pt:

   \assinatura{\textbf{\imprimirorientador} \\ Orientador} 
   \assinatura{\textbf{Professor} \\ Convidado 1}
   \assinatura{\textbf{Professor} \\ Convidado 2pt}
   %\assinatura{\textbf{Professor} \\ Convidado 3}
   %\assinatura{\textbf{Professor} \\ Convidado 4}
      
   \begin{center}
    \vspace*{0.5cm}
    {\large\imprimirlocal}
    \par
    {\large\imprimirdata}
    \vspace*{1cm}
  \end{center}

\end{comment}
%------------------------------------------------

%------------------------------------------------
% Dedicatória e Agradecimentos

%% ---
% Dedicatória
% ---
\begin{dedicatoria}
   \vspace*{\fill}
   \centering
   \noindent
   \textit{ Este trabalho é dedicado a ffffffs.} \vspace*{\fill}
\end{dedicatoria}
% ---

% ---
% Agradecimentos
% ---
\begin{agradecimentos}
Primeiramente a Deus, por ter me permitido entrar na UPE e num curso fantásti\-co, mesmo qu

\end{agradecimentos}
%------------------------------------------------

%------------------------------------------------
% Epígrafe

%\begin{epigrafe}
    \vspace*{\fill}
	\begin{flushright}
		\textit{`hhhhhh'\\ Isaac Newton}
	\end{flushright}
\end{epigrafe}
%------------------------------------------------

%------------------------------------------------
% RESUMOS

%% resumo em português
\begin{resumo}

O transporte público no Brasil sempre foi alvo de muitas reclamações ao longo do tempo. Na maioria das vezes, as queixas referem-se ao fato de os veículos estarem sempre lotados, às condições ruins dos carros e à baixa qualidade dos serviços prestados.

 \vspace{\onelineskip}
    
 \noindent
 \textbf{Palavras-chaves}: Computação em Nuvem, Cidades Criativas, Reconstrução 3D, Virtualiza\-ção, Avaliação de Desempenho
\end{resumo}





% resumo em inglês
\begin{resumo}[Abstract]
 \begin{otherlanguage*}{english}
   The Creative Cities movement search creative solutions from environmental issues or inventive ways for living people. Among the 
    
   \vspace{\onelineskip}
 
   \noindent 
   \textbf{Key-words}: Cloud Computing, creative cities, 3D reconstruction, virtualization, performance evaluation.
 \end{otherlanguage*}
\end{resumo}

\begin{comment}




% resumo em francês 
\begin{resumo}[Résumé]
 \begin{otherlanguage*}{french}
    Il s'agit d'un résumé en français.
 
   \vspace{\onelineskip}
 
   \noindent
   \textbf{Mots-clés}: latex. abntex. publication de textes.
 \end{otherlanguage*}
\end{resumo}




% resumo em espanhol
\begin{resumo}[Resumen]
 \begin{otherlanguage*}{spanish}
   Este es el resumen en español.
  
   \vspace{\onelineskip}
 
   \noindent
   \textbf{Palabras clave}: latex. abntex. publicación de textos.
 \end{otherlanguage*}
\end{resumo}

\end{comment}
%------------------------------------------------

%------------------------------------------------% inserir lista de ilustrações

%\pdfbookmark[0]{\listfigurename}{lof}
\listoffigures*
\cleardoublepage
%------------------------------------------------

%------------------------------------------------
% inserir lista de tabelas

%\pdfbookmark[0]{\listtablename}{lot}
\listoftables*
\cleardoublepage
%------------------------------------------------

%------------------------------------------------% inserir lista de abreviaturas e siglas

%\begin{siglas}
  \item[Fig.] Area of the $i^{th}$ component
  \item[456] Isto é um número
  \item[123] Isto é outro número
  \item[lauro cesar] este é o meu nome
\end{siglas}
%------------------------------------------------

%------------------------------------------------% inserir lista de símbolos

%\begin{simbolos}
  \item[$ \Gamma $] Letra grega Gama
  \item[$ \Lambda $] Lambda
  \item[$ \zeta $] Letra grega minúscula zeta
  \item[$ \in $] Pertence
\end{simbolos}
%------------------------------------------------

%------------------------------------------------% inserir o sumario

\pdfbookmark[0]{\contentsname}{toc}
\tableofcontents*
\cleardoublepage
%------------------------------------------------


% ----------------------------------------------------------
% ELEMENTOS TEXTUAIS
% ----------------------------------------------------------
\textual

% ----------------------------------------------------------
% Introdução
% ----------------------------------------------------------
\chapter[Introdução]{Introdução}
\addcontentsline{toc}{chapter}{Introdução}

Um aplicativo de geolocalização projetado para ser utilizado em ambientes abertos é classificado   como   sistema   de   posicionamento   outdoor,   ou   seja,   aplicações   de geolocalização para serem utilizadas fora de construções [Carrasco-Letelier, 2015], cujo sensor mais popular é o GPS (Global Positioning System). Já nos sistemas de posicionamento indoor, o objetivo é conhecer a localização ou trajetória do usuário quando este se encontra em um ambiente fechado, por exemplo,dentro   de   uma   construção. Existem   vários   exemplos   de   aplicações   de sistemas de localização indoor. Um desses é, por exemplo, se um dos pacientes em um local pra idosos fica duas horas no banheiro, os médicos podem ser notificados de que algo está errado \cite{silvasistema}.

Para as aplicações indoor o sensor GPS não é tão adequado, uma vez que o sinal é atenuado e disperso por telhados, paredes e outros objetos. Dessa forma, são utilizadas outras técnicas, que vão desde o uso de ondas de rádio a campos magnéticos e sinais acústicos \cite{curran2011evaluation}. Uma possível solução seria o uso de antenas Bluetooth, como apresentado por Bekkelien (2012): várias antenas são espalhadas por um edifício e a posição exata de cada uma deve ser registrada para que o sistema utilize esta posição como referência. A força do sinal é usada para calcular a distância do dispositivo para a antena   Bluetooth,   inferindo   assim   a   posição   do   usuário.   Um   obstáculo   na   solução descrita é a necessidade de dispositivos externos espalhados, que elevam o custo de implantação de acordo com o tamanho da área a ser coberta.

Outras   abordagens   são   encontradas   como   a   utilização   de   antenas   de   wi-fi \cite{liu2007survey}, e a utilização de transmissores de rádio frequência (RFID -  Radio-Frequency IDentification ) defendidas por Fabini, Russ e Wallentin   (2013),   e   Randell   e   Muller   (2001).   Porém   essas   técnicas   também apresentam os problemas levantados quanto à aquisição de equipamentos. Uma solução para   este   problema   é   o   uso   de   sensores   presentes   em   dispositivos   móveis   como celulares, como proposto por Stockx, Hecht e Schöning (2014). Este trabalho apresenta uma implementação para o problema de posicionamento indoor,   capaz   de   fornecer   tanto   a   posição   de   um   usuário   quanto   sua   trajetória   em ambientes internos. Para isso são usados os sensores existentes nos smartphones e técnicas com Wi-Fi, já que é a tecnologia RF mais difundida e estudada pelos pesquisadores. A Identificação da Força do Sinal Recebida (IFSR) é conhecida como um recurso de propagação de sinais Wi-Fi \cite{ohara2017metric}.
Métodos usando RSSI como um radar para localização interna estimam a distância entre transmissores com coordenadas conhecidas e um receptor baseado em valores de IFSR e um modelo de atenuação de rádio, e, em seguida, localiza o receptor com base no princípio triangulação.




\section{Problema} 

Com o aumento do número de pessoas nos centros comerciais se viu necessário o uso de tecnologias que possibilitassem aos varejistas obterem algum tipo de informação que os ajudassem a melhorar o seu negócio, pois, partir do momento em que é possível impactar o consumidor no momento mais apropriado, a performance e qualidade do serviço são otimizados. A tecnologia de geolocalização simples (GPS) deixa muito a desejar, pois possui muita tolerância a erros de leitura do sensor. A Geolocalização Interna pode se utilizar em diferentes nichos do mercado. No marketing, esta tecnologia é útil para entregar conteúdo relevante, cupons de desconto, promoções e anúncios personalizados de acordo com os hábitos de consumo da audiência.

\section{Hipóteses}

Através de técnicas de teoria dos grafos e triangularização, uma das hipóteses é que será possível verificar a disposição de pessoas que estão em um ambiente e também a movimentação de cada uma delas, com essa análise pode-se criar técnicas de planejamento e publicidade direcionada. 

\section{Objetivos} 

O objetivo dessa monografia é de estudar técnicas de triangularização de posição para através desses estudos criar uma plataforma que gere dados ao consumidor lhe mostrando que o que ele quer e/ou precisa está a poucos metros de distância.

\section{Justificativa} 

A tecnologia no varejo tem um papel fundamental para que varejistas consigam administrar seus negócios, analisar os resultados de seus esforços em diversas áreas, atrair clientes, e por fim, engajar com eles. Na prática isso significa que o varejo brasileiro pode explorar bastante novas soluções tecnológicas disponíveis para os smartphones brasileiros.



% Reconstrução 3D
% ----------------------------------------------------------
%\input{elementos-textuais/reconstrucao}

% ----------------------------------------------------------


% ---
% Capitulo de revisão de literatura
% ---
% ---
%\chapter[Revisão da Literatura]{Revisão da Literatura}
\addcontentsline{toc}{chapter}{Revisão da Literatura}

\setcounter{footnote}{6}



% ---

% Trabalhos Relacionados
% ----------------------------------------------------------
%\begin{comment}

\chapter[Trabalhos-Relacionados]{Trabalhos Relacionados}
\addcontentsline{toc}{chapter}{Trabalhos-Relacionados}

\end{comment}

\section{Trabalhos Relacionados}
\label{sec-trabalhos-relacionados}



% ---


% PARTE - Metodologia
% ----------------------------------------------------------
%%\part{Metodologia}



\chapter[Metodologia]{Metodologia}
\addcontentsline{toc}{chapter}{Metodologia}
\label{sec:experimental-design}

\setcounter{footnote}{23}


Os principais objetivos desta avaliação de desempenho são: (a) determinar o \textit{overhead} da ferramenta Bundler em ambientes virtualizados; (b) verificar qual recurso é mais crucial no ambiente virtualizado, CPU, memória RAM e/ou tempo de processamento; e (c) verificar o ganho em migrar a aplicação para o ambiente de nuvem.

\subsection{Cenários}
\label{subsec:cenarios}

Neste trabalho foram feitas avaliações em dois cenários. O primeiro cenário consiste em uma máquina com configurações simples, e o segundo  O LXC foi utilizado pois utiliza recursos do Kernel do Linux para criar os \textit{containers} \cite{dua2014virtualization}.

\subsection{Metodologia}
\label{subsec:metodologia}

Para alcançar os objetivos definidos neste trabalho, adaptou-se a metodologia utilizada por \cite{sousa2012evaluating} para executar os experimentos, baseado em quatro principais atividades (Figura \ref{fig:methodology}):


\begin{figure}[h!]
\centering
\includegraphics[width=1\columnwidth]{imagens/metodologia.png}
\caption{Metodologia de execução dos experimentos (adaptado de \cite{sousa2012evaluating})}
\label{fig:methodology}
\end{figure}




% ----------------------------------------------------------
% Capitulo com exemplos de comandos inseridos de arquivo externo 
% ----------------------------------------------------------

%\include{abntex2-modelo-include-comandos}




% ----------------------------------------------------------
% Resultados
% ----------------------------------------------------------
%\chapter[Resultados]{Resultados}
\addcontentsline{toc}{chapter}{Resultados}

Este capítulo mostra os resultados obtidos dos experimentos realizados. Primeiro os resultados obtidos do primeiro cenário serão apresentados, depois os resultados obtidos do segundo cenário e, por fim, é apresentada uma discussão acerca dos resultados obtidos deste trabalho.

\section{Primeiro Cenário}

\ref{tabela-tempo-execucao-nativo-cenario1} apresenta o tempo necessário para se processar conjuntos com quantidades diferentes de imagens (5, 9, 13 e 130) utilizando a infraestrutura de \textit{hardware} para o primeiro cenário, apresentada na subse\-ção \ref{sec:infra}, sem virtualização. 





\begin{table}[h]
\centering
\caption{Tempo de execução da aplicação Bundler no ambiente não-virtualizado}
\label{tabela-tempo-execucao-nativo-cenario1}
\begin{tabular}{|l|l|l}
\cline{1-2}
Quantidade de imagens & Tempo de execução (em minutos) &  \\ \cline{1-2}
5                 & 00:14                          &  \\ \cline{1-2}
9                 & 00:28                          &  \\ \cline{1-2}
13                & 00:43                          &  \\ \cline{1-2}
130               & 34:13                          &  \\ \cline{1-2}
\end{tabular}
\end{table}




% Finaliza a parte no bookmark do PDF, para que se inicie o bookmark na raiz
% ---
%\bookmarksetup{startatroot}% 
% ---

% ---
% Conclusão
% ---
%\chapter[Conclusão]{Conclusão}
\addcontentsline{toc}{chapter}{Conclusão}



% ----------------------------------------------------------
% ELEMENTOS PÓS-TEXTUAIS
% ----------------------------------------------------------
%\postextual

%------------------------------------------------
% Referências
\bibliography {abntex2-modelo-references}

%------------------------------------------------

%------------------------------------------------
% Glossário
%% ----------------------------------------------------------
% Glossário
% ----------------------------------------------------------
%
% Consulte o manual da classe abntex2 para orientações sobre o glossário.
%
%\glossary
%------------------------------------------------


%------------------------------------------------
% Apêndices
%
\begin{apendicesenv}

\partapendices

\chapter{Artigo}
\label{ap:artigo}

\end{apendicesenv}



\begin{comment}


% ---
% Inicia os apêndices
% ---
\begin{apendicesenv}

% Imprime uma página indicando o início dos apêndices
\partapendices
%\includepdf[pages=-]{artigo.pdf}
% ----------------------------------------------------------
\chapter{Quisque libero justo}
% ----------------------------------------------------------

\lipsum[50]

% ----------------------------------------------------------
\chapter{Nullam elementum urna vel imperdiet sodales elit ipsum pharetra ligula
ac pretium ante justo a nulla curabitur tristique arcu eu metus}
% ----------------------------------------------------------
\lipsum[55-57]

\end{apendicesenv}
% ---
\end{comment}


%\includepdf[pages=-]{artigo.pdf}
%------------------------------------------------


%------------------------------------------------
% Anexos
%% ----------------------------------------------------------
% Anexos
% ----------------------------------------------------------

%\begin{anexosenv}
%\partanexos
%lalalla
%\end{anexosenv}

%\includepdf[pages=-]{IEEE_latin_2016_guto (1)-2-11.pdf}

\begin{comment}

%inserindo apenas uma página
\includepdf[pages=16]{arquivo01.pdf}

%inserindo algumas páginas: 1 até 7, depois 50 e 57
\includepdf[pages={1-7,50,57}]{arquivo02.pdf}

%inserindo uma página em branco depois da página 1
\includepdf[pages={1,{},2-10}]{arquivo03.pdf}

%inserindo todas as páginas
\includepdf[pages=-]{arquivo04.pdf}

%inserindo múltiplas páginas numa única página
\includepdf[pages={286-291},nup=2x3]{arquivo05.pdf}

%inserindo páginas em landscape
\includepdf[pages=-,landscape]{arquivo06.pdf}


% ---
% Inicia os anexos
% ---
\begin{anexosenv}

% Imprime uma página indicando o início dos anexos
\partanexos

% ---
\chapter{Morbi ultrices rutrum lorem.}
% ---
\lipsum[30]

% ---
\chapter{Cras non urna sed feugiat cum sociis natoque penatibus et magnis dis
parturient montes nascetur ridiculus mus}
% ---

\lipsum[31]

% ---
\chapter{Fusce facilisis lacinia dui}
% ---

\lipsum[32]

\end{anexosenv}

\end{comment}
%------------------------------------------------

%------------------------------------------------
% INDICE REMISSIVO
\printindex
%------------------------------------------------



\end{document}
